\documentclass[ngerman]{dis-template-add}


\renewcommand{\Aufgabenblatt}{1}
\renewcommand{\Ausgabedatum}{28. April 2020}
\renewcommand{\Abgabedatum}{8. Mai 2020}
\renewcommand{\Gruppe}{Simon Weidmann, Aram Yesildeniz}
\renewcommand{\STiNEGruppe}{14}


\begin{document}


\section{Beispiel für ER-Diagramm}


\begin{center}
\begin{tikzpicture}
\node[entity] (e1) {E1};
\node[attribut] (e1-a1) [above left =5mm and 4mm of e1.north] {\underline{A1}} edge (e1);
\node[attribut] (e1-a2)  [above right=5mm and 4mm of e1.north] {A2} edge (e1);
\node[multivalentattribut] (e1-a3)  [right=5mm of e1] {A3} edge (e1);

\node[is-a] (is-a) [below =1cm of e1.south]              {is-a} edge [erbt] (e1);
% \node[is-a-blue] statt \node[is-a] für ein blau eingefärbtes is-a-Element
% Bsp: \node[is-a-blue] (is-a) [below =1cm of e1.south] {is-a} edge [erbt] (e1);

\node[entity] (e2) [below left =1cm and 1mm of is-a.south] {E1} edge [erbt] (is-a);
\node[entity] (e3) [below right=1cm and 1mm of is-a.south] {E3} edge [erbt] (is-a);
\node[entity] (e4) [right =7cm of e3] {E4};

\node[weakentity] (e5) [below =3cm of e3] {E5};
\node[attribut] (e5-a1)  [right=5mm of e5] {\dashuline{A1}} edge (e5);

\node[relationship] (r1) [right=2cm of e3] {R1};
\path (r1) edge node[at end,anchor=north west] {$[0;2]$} (e3);
\path (r1) edge node[at end,anchor=north east] {$[7;9]$} (e4);


\node[weakrelationship] (r2) [below=1cm of e3] {R2};
\path (r2) edge node[at end,anchor=north west] {$1$} (e3);
\draw[double distance=2pt] (r2) -- node[at end,anchor=south east] {$8$} (e5);

% reflexive relationship types
\node[relationship] (r4) [below=2cm of e4] {R4};
\path (r4.west) edge node[anchor=north east] {$[0;*]$} ([shift={(1.5em,0em)}] e4.south west);
\path (r4.east) edge node[anchor=north west] {$[1;1]$} ([shift={(-1.5em,0em)}] e4.south east);
\end{tikzpicture}
\end{center}






\section{Beispiel für relationales Datenbankschema}

\begin{RMSchma}
Person(\soliduline{PID}, Name, Vorname, \dashuline{(HaustierName, HaustierRasse) $\rightarrow$ (Haustier.Name, Haustier.Rasse)})

Haustier(\soliduline{Name, Rasse}, \dashuline{Herrchen $\rightarrow$ Person.PID})
\end{RMSchma}




\section{Beispiel für Ausdruck der Relationenalgebra}

\begin{align*}
 &\umbenennung{Rasse}{Sorte}(\projektion{Rasse, Geschlecht}((Wolf\verbund{Wolf.WID=Haustier.HID} (\selektion{Name=\wert{Hasso}}Haustiere)) \natverbund Person))
\\  &=\{ \wert{Steppenwolf}, \wert{m} \}
\end{align*}




\newpage
\section{Beispiel für SQL-Anfrage}

\begin{verbatim}
SELECT 
  h.Name,
  h.Rasse
FROM 
  Haustier h,
  Person p
WHERE
  h.Herrchen = p.PID AND
  p.Vorname LIKE "P%"
\end{verbatim}








\section{Beispiel für Operatorbaum}

\begin{tikzpicture}
\node (Haustier) {Haustier};
\node (Wolf) [left=25mm of Haustier] {Wolf};
\node (join1) [above=20mm of $(Haustier)!.5!(Wolf)$] {$\verbund{Wolf.WID=Haustier.HID}$};
\node (selektion1) [above=of join1] {$\selektion{Name=\wert{Hasso}}$};
\node (projektion) [above=of selektion1] {$\projektion{Rasse}$};
\node (final) [above=of projektion] {};

\path (Haustier) edge node[smallr,near start,above right] {200 Tupel\\4 Attribute} (join1);
\path (Wolf) edge node[smalll,near start,above left] {1000 Tupel\\6 Attribute} (join1);
\path (join1) edge node[smallr,near start,above left] {?? Tupel\\?? Attribute} (selektion1);
\path (selektion1) edge node[smallr,midway,left] {$??\cdot\frac{??}{??}=??$ Tupel\\?? Attribute} (projektion);
\path (projektion) edge node[smallr,midway,left] {$??$ Tupel\\1 Attribut} (final);
\end{tikzpicture}







\section{Beispiel fürr Tabelle mit Sperranforderungen}

\begin{tabular}{|p{2cm}|p{2cm}|p{2cm}|p{2cm}|p{1cm}|p{1cm}|p{1cm}|p{3cm}|}
\hline
Zeitschritt & T\ts{1} & T\ts{2} & T\ts{3} & x & y & z & Bemerkung\\
\hline
\hline
0 &  &  &  & NL & NL & NL & \\
\hline
1 & lock(x,X) &  &  & X\ts{1} & NL & NL & \\
\hline
2 & write(x) & lock(y,R) &  & X\ts{1} & R\ts{2} & NL & \\
\hline
3 &  &  &  &  &  &  & \\
\hline
4 &  &  &  &  &  &  & \\
\hline
5 &  &  &  &  &  &  & \\
\hline
\end{tabular}



\section{Beispiel für B-  und B*-Bäumen}

Löschen Sie aus dem unten abgebildeten  \textbf{B*-Baum} der Klasse $\tau(1,2,h)$
die Datensätze mit den Schlüsselwerten \textbf{40}, \textbf{43}, \textbf{38}, \textbf{32} und \textbf{90} (in dieser Reihenfolge).
Geben Sie jeweils kurz an, welche konkrete Maßnahme Sie durchgeführt haben (Mischen, Ausgleichen, einfaches Löschen) und zeichnen Sie den Baum nach jedem Mischen und Ausgleichen neu.
Für Ausgleichs- und Mischoperationen sollen nur direkt benachbarte Geschwisterknoten (bevorzugt der rechte) herangezogen werden.

\begin{center}
\begin{tikzpicture}
\tikzstyle{bplus}=[rectangle split, rectangle split horizontal,rectangle split ignore empty parts,draw]
\tikzstyle{every node}=[bplus]
\tikzstyle{level 1}=[sibling distance=30mm]
\tikzstyle{level 2}=[sibling distance=15mm]

\node {12 \nodepart{two} 47  } [->]
 child {node {6 \nodepart{two} 12 \nodepart{three} \;\;\; \nodepart{four} \;\;\; } }
 child {node {32 \nodepart{two} 38 \nodepart{three} 40 \nodepart{four} 43}}
  child {node {86 \nodepart{two} 88 \nodepart{three} 90 \nodepart{four} \;\;\;}}
;\end{tikzpicture}
\end{center}


40 und 43, Einfaches Löschen\\
38, Ausgleichen

\begin{center}
\begin{tikzpicture}
\tikzstyle{bplus}=[rectangle split, rectangle split horizontal,rectangle split ignore empty parts,draw]
\tikzstyle{every node}=[bplus]
\tikzstyle{level 1}=[sibling distance=30mm]
\tikzstyle{level 2}=[sibling distance=15mm]

\node {12 \nodepart{two} 86 } [->]
 child {node {6 \nodepart{two} 12 \nodepart{three} \;\;\; \nodepart{four} \;\;\;}}
 child {node {32 \nodepart{two} 86 \nodepart{three} \;\;\; \nodepart{four} \;\;\;}}
  child {node {88 \nodepart{two} 90 \nodepart{three} \;\;\; \nodepart{four} \;\;\;}}
;\end{tikzpicture}
\end{center}

32, Mischen

\begin{center}
\begin{tikzpicture}
\tikzstyle{bplus}=[rectangle split, rectangle split horizontal,rectangle split ignore empty parts,draw]
\tikzstyle{every node}=[bplus]
\tikzstyle{level 1}=[sibling distance=30mm]
\tikzstyle{level 2}=[sibling distance=15mm]

\node {12} [->]
 child {node {6 \nodepart{two} 12 \nodepart{three} \;\;\; \nodepart{four} \;\;\;}}
 child {node {86 \nodepart{two} 88 \nodepart{three} 90 \nodepart{four} \;\;\;}}
;\end{tikzpicture}
\end{center}

90, Einfaches Löschen







\end{document}