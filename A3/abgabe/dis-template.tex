\documentclass[ngerman]{dis-template-add}


\renewcommand{\Aufgabenblatt}{3}
\renewcommand{\Ausgabedatum}{12. Mai 2020}
\renewcommand{\Abgabedatum}{19. Mai 2020}
\renewcommand{\Gruppe}{Simon Weidmann, Aram Yesildeniz}
\renewcommand{\STiNEGruppe}{14}


\begin{document}


\section*{3.1 Isolation Levels and SQL}

\subsection*{a)}

\textit{How can you determine the currently set isolation level?
}
\begin{verbatim}
SHOW TRANSACTION ISOLATION LEVEL;
\end{verbatim}

\textit{What is the default isolation level of PostgreSQL?
} \\
Read Committed is the default isolation level in PostgreSQL. \\

\textit{How can the isolation level be changed during a session in PostgreSQL?}

For current transaction:
\begin{verbatim}
SET TRANSACTION ISOLATION LEVEL
{ SERIALIZABLE | REPEATABLE READ | READ COMMITTED | READ UNCOMMITTED }
\end{verbatim}

For default transaction characteristics:
\begin{verbatim}
SET SESSION CHARACTERISTICS AS TRANSACTION ISOLATION LEVEL
{ SERIALIZABLE | REPEATABLE READ | READ COMMITTED | READ UNCOMMITTED }
\end{verbatim}


\subsection*{b)}

\begin{verbatim}
CREATE TABLE public."OPK"
(
    "ID" integer,
    "NAME" text
);

ALTER TABLE public."OPK"
    OWNER to postgres;
\end{verbatim}


\subsection*{c)}

\begin{verbatim}
INSERT INTO public."OPK"(
    "ID", "NAME")
    VALUES (1, 'shaggy');

INSERT INTO public."OPK"(
    "ID", "NAME")
    VALUES (2, 'fred');
	
INSERT INTO public."OPK"(
    "ID", "NAME")
    VALUES (3, 'velma');
	
INSERT INTO public."OPK"(
    "ID", "NAME")
    VALUES (4, 'scooby');
	
INSERT INTO public."OPK"(
    "ID", "NAME")
    VALUES (5, 'daphne');
\end{verbatim}


\subsection*{d)}

\textbf{Read Committed}

Read Committed is an isolation level.
Under read committed isolation, the user is only allowed to read committed data.
It is guaranteed that the query returns only data which was committed at the time of the read.
If a transaction needs to read a row that has been modified by an incomplete transaction in another session, the transaction waits until the first transaction completes (either commits or rolls back).
Therefore read committed isolation prevents Dirty Read.

Read committed holds exclusive locks during the whole transaction.
Shared locks on the other hand are acquired as late as possible and released as soon as possible.
Therefore, Phantom Reads and Non-Repeatable Reads can occur.

In DB2, the read committed isolation level is called \textit{Cursor Stability} and is the default isolation level.

Example:
In the example, the transaction would hold exclusive locks for all write or update queries. Since the transaction will only read data, no exclusive lock will be hold.
For the SELECT query, the transaction will hold a shared lock for the OPK table where ID is equal to 3. It will only read the data, if there are no uncommitted changes.
After the SELECT query and before the commit, the transaction will release the shared lock.
After the commit, no locks are hold by the query.


\subsection*{e)}




The transaction can repeat the same query, and no rows that have been read by the transaction will have been updated or deleted.



Under Repeatable Reads isolation level, Shared Locks are acquired for the transaction duration. "Dirty Reads" and "Non-Repeatable Reads" are prevented but "Phantom Reads" can still occur.

REPEATABLE READ guarantees that no item you've selected the first time can be modified or deleted until you commit.


%\begin{RMSchma}
%Person(\soliduline{PID}, Name, Vorname, \dashuline{(HaustierName, HaustierRasse) $\rightarrow$ (Haustier.Name, Haustier.Rasse)})
%Haustier(\soliduline{Name, Rasse}, \dashuline{Herrchen $\rightarrow$ Person.PID})
%\end{RMSchma}

%\section{Beispiel für Ausdruck der Relationenalgebra}

%\begin{align*}
% &\umbenennung{Rasse}{Sorte}(\projektion{Rasse, Geschlecht}((Wolf\verbund{Wolf.WID=Haustier.HID} (\selektion{Name=\wert{Hasso}}Haustiere)) \natverbund Person))
%\\  &=\{ \wert{Steppenwolf}, \wert{m} \}
%\end{align*}


\section*{3.2 Lock Conflicts}

\subsection*{a)}

\subsection*{b)}

\subsection*{c)}

\subsection*{d)}

\subsection*{e)}


\end{document}